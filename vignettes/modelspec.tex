\documentclass{article}
\usepackage{amsmath}

%\VignetteIndexEntry{Specifying Models} 
\title{Specifying Models and Parameter Starting Values}
\author{David Borchers}
\usepackage{Sweave}
\begin{document}
\maketitle
\Sconcordance{concordance:modelspec.tex:/home/theo/git/hsltm/vignettes/modelspec.Rnw:%
1 6 1 1 0 116 1}



You have to pass function \verb|est.hmltm| the name of the discrete detection hazard function $h(,y)$ to use (via the argument \verb|FUN|) a model specification (via the argument \verb|model|) and a vector of starting parameters (via the argument \verb|pars|). The possible forms for $h(x,y)$ are shown in Table~\ref{tab:hazfuns}, together with their scale and shape parameters (each form has one or more shape paramters and one or more shape parameters). 

\begin{itemize}
\item[{\tt FUN}] is a character variable containing the name of the detection hazard model $h(x,y)$. Valid names are shown in the first columnn of Table~\ref{tab:hazfuns}.
\item[{\tt model}] is a list containing two character variables named \verb|y| and \verb|x| that specify the form of the dependence of the model's scale parameter(s) on covariates (or \verb|NULL| if there are no covariates). The syntax of the model specification is given below (it is similar to \verb|R|'s usual regression model syntax.)
\item[{\tt pars}] is a numeric vector specifying starting values for the model parameters. The parameter starting values obviously depend on \verb|model| and on which, if any, covariates you're using. 
\end{itemize}

\section{Covariates and parameter starting values}

Covariates always affect the scale parameter, not the shape parameter. The reason for this is historical: line transect estimators tend to be implemented with covariates affecting the scale parameter. The only rationale I have ever found for this is a line transect study of beer cans reported in Pollock et al. (1990) in which having covariates affect the scale parameter of detection functions was found to be more effective than having them affect the shape parameter.

\begin{table}
\centering
\caption{Forms and parameters of the discrete hazard function, $h(x,y)$. Here $x$ and $y$ are perpendicular and forward distances, respectively, $\mathbf{X}$ and $\mathbf{X}_x$ are design matrices for covariates affecting the scale parameters and $\boldsymbol{\beta}$ and $\boldsymbol{\beta}_x$ are parameter vectors associated with the covariates affecting the scale parameters. 
\label{tab:hazfuns}}
\begin{tabular}{lcccc}
\hline
       &     &Shape        & \multicolumn{2}{c}{Scale parameter(s)}\\
\cline{4-5}
Name   &Form &parameter(s) &No covariates &With covariates \\
\hline \\
h.IP.0 &$\sigma^\gamma/\left(\sigma^2+x^2+y^2\right)^{\gamma/2}$ & $\gamma$ & $\sigma$ & $\sigma\exp(\mathbf{X}\boldsymbol{\beta})$
\vspace{10pt} \\
h.EP1.0 &$\exp\left\{-\left(\frac{x}{\sigma}\right)^{\gamma}-\left(\frac{y}{\sigma}\right)^{\gamma}\right\}$ & $\gamma$ & $\sigma$ & $\sigma\exp(\mathbf{X}\boldsymbol{\beta})$
\vspace{10pt} \\
h.EP2.0 &$\exp\left\{-\left(\frac{x}{\sigma}\right)^{\gamma_x}-\left(\frac{y}{\sigma}\right)^{\gamma_y}\right\}$ & $\gamma_x$, $\gamma_y$ & $\sigma$ & $\sigma\exp(\mathbf{X}\boldsymbol{\beta})$
\vspace{10pt} \\
h.EP1x.0 &$\exp\left\{-\left(\frac{x}{\sigma_x}\right)^{\gamma}-\left(\frac{y}{\sigma}\right)^{\gamma}\right\}$ & $\gamma$ & $\sigma$, $\sigma_x$ & $\sigma\exp(\mathbf{X}\boldsymbol{\beta})$, $\sigma_x\exp(\mathbf{X}_x\boldsymbol{\beta}_x)$  
\vspace{10pt} \\ 
h.EP2x.0 &$\exp\left\{-\left(\frac{x}{\sigma_x}\right)^{\gamma_x}-\left(\frac{y}{\sigma}\right)^{\gamma}\right\}$ & $\gamma$, $\gamma_x$& $\sigma$, $\sigma_x$ & $\sigma\exp(\mathbf{X}\boldsymbol{\beta})$, $\sigma_x\exp(\mathbf{X}_x\boldsymbol{\beta}_x)$  \\
\\
\hline
\end{tabular}
\end{table}

Deciding on starting parameter values involves some trial and error. A reasonable starting value for $\sigma$ is to make it similar size to the average $x$ and $y$ (when the model does not include $\sigma_x$), average $y$ (when the model includes $\sigma_x$). A reasonable value for $\sigma_x$ is something around the size of the average $x$. Shape parameter values between about 0.5 and 2 have worked for cases I've looked at, so a value around 1 might be a reasonable starting value guess.

When you've got covariates you also need to specify starting values for the parameters associated with the covariates. You do this by specifying starting values for $\exp(\beta)$ for each $\beta$ in the vector $\boldsymbol{\beta}$. Some guidance can be got from the facts that

\begin{description}
\item[$\exp(\beta)=1$] implies the covariate has no effect,
\item[$0<\exp(\beta)<1$] implies that increasing values of the covariate decrease $\sigma$,
\item[$1<\exp(\beta)$] implies that increasing values of the covariate increase $\sigma$.
\end{description}

In the absence of information about the effect of the covariate on $\sigma$ a starting value of $\exp(\beta)=1$ seems sensible. 

If we let $e^{\beta_{1}},\ldots,e^{\beta_{m}}$ be the starting values for the $m$ covariates affecting $\sigma$ and $e^{\beta_{x1}},\ldots,e^{\beta_{xm_x}}$ be the starting values for the $m_x$ covariates affecting $\sigma_x$, then the starting values are put in a vector (named \verb|pars| here) in the order shown below, omitting values for any parameters that are not relevant for the hazard function and model being used:\\
\verb|pars=c(|$\gamma$,$\gamma_x$,$\sigma$,$e^{\beta_{1}},\ldots,e^{\beta_{m}}$, $\sigma_x$,$e^{\beta_{x1}},\ldots,e^{\beta_{xm_x}}$\verb|)|. 



\section{Some example starting values for covariates}

Model h.IP.0 with no covariates and starting values $\gamma=1$, $\sigma=10$:
\begin{verbatim}
hfun="h.IP.0"
models=list(y=NULL,x=NULL)
pars=c(1,10)
\end{verbatim}

\noindent
Model h.IP.0 with covariates \verb|bf| and \verb|cu| affecting $\sigma$, and starting values $\gamma=3.4$, $\sigma=54$, $e^{\beta_{bf}}=0.7$, $e^{\beta_{cu}}=0.9$ (Note that when there is no $\sigma_x$ parameter, the dependence of $\sigma$ is specified in the \verb|model| via element \verb|y=...|.):
\begin{verbatim}
hfun="h.IP.0"
models=list(y="bf+cu",x=NULL)
pars=c(3.4, 54, 0.7, 0.9)
\end{verbatim}


\noindent
Model h.EP2x.0 with no covariates and starting values $\gamma_x=0.76$, $\gamma_y=0.86$, $\sigma=19.2$, $\sigma_x=13.1$:
\begin{verbatim}
hfun="h.EP2x.0"
models=list(y=NULL,x=NULL)
pars=c(0.76, 0.86, 19.2, 13.1)
\end{verbatim}

\noindent
Model h.EP2x.0 with a single covariate (\verb|bf|) affecting only $\sigma_x$, and starting values $\gamma_x=0.62$, $\gamma_y=0.96$, $\sigma=24.1$, $\sigma_x=17.5$, $e^{\beta_{x,bf}}=0.65$:
\begin{verbatim}
hfun="h.EP2x.0"
models=list(y=NULL,x="~bf")
pars=c(0.62, 0.96, 24.1, 17.5, 0.65)
\end{verbatim}

\noindent
Model h.EP2x.0 with a single covariate (\verb|bf|) affecting only $\sigma$, and starting values $\gamma_x=1.1$, $\gamma_y=0.8$, $\sigma=30$, $e^{\beta_{bf}}=0.7$, $\sigma_x=27$:
\begin{verbatim}
hfun="h.EP2x.0"
models=list(y="~bf",x=NULL)
pars=c(1.1, 0.8, 30, 0.7, 27)
\end{verbatim}

\noindent
Model h.EP2x.0 with covariate \verb|bf| affecting $\sigma$ and covarites \verb|bf| and \verb|cu| affecting $\sigma_x$, and starting values $\gamma_x=1.1$, $\gamma_y=0.8$, $\sigma=30$, $e^{\beta_{bf}}=0.7$, $\sigma_x=27$, $e^{\beta_{x,bf}}=0.6$, $e^{\beta_{x_cu}}=0.9$:
\begin{verbatim}
hfun="h.EP2x.0"
models=list(y="~bf",x="~bf+cu")
pars=c(1.1, 0.8, 30, 0.7, 27, 0.6, 0.9)
\end{verbatim}

\section{Some example starting values for factors}

Note done this yet...

\section*{References}

Pollock, K.H. and Otto, M.C. 1990. Size bias in line transect sampling: a field test. {\it Biometrics} {\bf 46}: 239-245.

\end{document}
